\documentclass[nobib]{tufte-handout}


\usepackage{ifluatex, ifxetex} % проверка, как компилируется файл

% black magic!
% solving problem with tufte-handout + xelatex
% http://tex.stackexchange.com/questions/200722/
% https://tex.stackexchange.com/questions/47576/
\ifx\ifxetex\ifluatex\else % если xelatex или lualatex
  \newcommand{\textls}[2][5]{%
    \begingroup\addfontfeatures{LetterSpace=#1}#2\endgroup
  }
  \renewcommand{\allcapsspacing}[1]{\textls[15]{#1}}
  \renewcommand{\smallcapsspacing}[1]{\textls[10]{#1}}
  \renewcommand{\allcaps}[1]{\textls[15]{\MakeTextUppercase{#1}}}
  \renewcommand{\smallcaps}[1]{\smallcapsspacing{\scshape\MakeTextLowercase{#1}}}
  \renewcommand{\textsc}[1]{\smallcapsspacing{\textsmallcaps{#1}}}
\fi

% с альтернативного источника:
%\ifluatex % Allow rendering with LuaTeX by setting up the spacing using fontspec features
%  \renewcommand\allcapsspacing[1]{{\addfontfeature{LetterSpace=15}#1}}
%  \renewcommand\smallcapsspacing[1]{{\addfontfeature{LetterSpace=10}#1}}
%\fi

% lua produces strange notes
% https://tex.stackexchange.com/questions/328431

\usepackage{fontspec} % работа со шрифтами
\usepackage{polyglossia} % учим русский как иностранный :)

\setmainlanguage{russian}
\setotherlanguages{english}

% заменяем --- на тире, << на кавычки и т.д.:
\defaultfontfeatures{Ligatures = TeX}

% download "Linux Libertine" fonts:
% http://www.linuxlibertine.org/index.php?id=91&L=1
\setmainfont{Linux Libertine O} % or Helvetica, Arial, Cambria
% why do we need \newfontfamily:
% http://tex.stackexchange.com/questions/91507/
\newfontfamily{\cyrillicfonttt}{Linux Libertine O}
\newfontfamily{\cyrillicfont}{Linux Libertine O}
\newfontfamily{\cyrillicfontsf}{Linux Libertine O}



\usepackage{amsmath}
\usepackage{amsthm}
\usepackage{amsfonts}
\usepackage{amssymb}
\usepackage{url} % вставка \url{}
\usepackage{graphicx} % вставка графиков
\usepackage{csquotes} % адаптирующиеся кавычки командой \enquote{}
\usepackage{comment} % ingore everything between \begin{comment} \end{comment}
\usepackage{answers} % separate problems and solutions
\usepackage{tikz} % pictures with tikz language
\usepackage{todonotes} % todo in documents


\usepackage{enumitem} % для создания своих нумерующих списков (хак для гиперссылок)

%\usepackage[left=2cm,right=2cm,top=2cm,bottom=2cm]{geometry}

\theoremstyle{definition}
% \newtheorem{problem}{Задача}
% \numberwithin{problem}{section}

\Newassociation{sol}{solution}{solution_file}
% sol — имя окружения внутри задач
% solution — имя окружения внутри solution_file
% solution_file — имя файла в который будет идти запись решений
% можно изменить далее по ходу


% very useful during de-bugging!
% \usepackage[left]{showlabels}
% \showlabels{hypertarget}
% \showlabels{hyperlink}

% магия для автоматических гиперссылок задача-решение
\newlist{myenum}{enumerate}{3}
% \newcounter{problem}[chapter] % нумерация задач внутри глав
\newcounter{problem}

\newenvironment{problem}%
{%
\refstepcounter{problem}%
%  hyperlink to solution
     % \hypertarget{problem:{\thechapter.\theproblem}}{} % нумерация внутри глав
     \hypertarget{problem:{\theproblem}}{}
     %\Writetofile{solution_file}{\protect\hypertarget{soln:\thechapter.\theproblem}{}}
     \Writetofile{solution_file}{\protect\hypertarget{soln:\theproblem}{}}
     % \begin{myenum}[label=\bfseries\protect\hyperlink{soln:\thechapter.\theproblem}{\thechapter.\theproblem},ref=\thechapter.\theproblem]
     \begin{myenum}[label=\bfseries\protect\hyperlink{soln:\theproblem}{\theproblem},ref=\theproblem]
     \item%
    }%
    {%
    \end{myenum}}
% для гиперссылок обратно надо переопределять окружение
% это происходит непосредственно перед подключением файла с решениями


\theoremstyle{definition}
\newtheorem{definition}{Определение}



\DeclareMathOperator{\Var}{Var}
\DeclareMathOperator{\card}{card}
\DeclareMathOperator{\E}{\mathbb{E}}
\newcommand{\I}{\mathbb{I}} % индикатор события
%% эконометрические сокращения
\DeclareMathOperator{\Cov}{Cov}
\DeclareMathOperator{\Arg}{Arg}
\DeclareMathOperator{\Corr}{Corr}
\def \hb{\hat{\beta}}
\def \hs{\hat{\sigma}}
\def \htheta{\hat{\theta}}
\def \s{\sigma}
\def \hy{\hat{y}}
\def \hY{\hat{Y}}
\def \v1{\vec{1}}
\def \e{\varepsilon}
\def \he{\hat{\e}}
\def \z{z}
\def \hVar{\widehat{\Var}}
\def \hCorr{\widehat{\Corr}}
\def \hCov{\widehat{\Cov}}
\def \cN{\mathcal{N}}
\let\P\relax
\DeclareMathOperator{\P}{\mathbb{P}}



\usepackage[bibencoding = auto,
backend = biber,
sorting = nyt, % name-year-title sorting
style=authoryear]{biblatex}

\addbibresource{ts_intro_book.bib}

\AddEnumerateCounter{\asbuk}{\russian@alph}{щ} % для списков с русскими буквами
\setlist[enumerate, 1]{label=\asbuk*),ref=\asbuk*}



\title{Учебник по временным рядам: начало}
\author{Винни-Пух}
\date{\today}

\begin{document}



\maketitle

\part{Одномерные временные ряды}


\chapter{Стационарные процессы}


\Opensolutionfile{solution_file} % [sols_chap_07]
% в квадратных скобках можно уточнить фактическое имя файла


Из курса математического анализа мы знаем разницу между рядами и последовательностями. 
В последовательности числа записаны одно за другим, скажем, через запятую,

\[
5, 8, -3, 2, 4, 5, \ldots  
\]

А ряд — это бесконечная сумма чисел, например,

\[
0.9 + 0.09 + 0.009 + 0.0009 + \ldots  
\]

Настала пора дать первое определение и раскрыть глаза!

\begin{definition}
Временной ряд — это последовательность\marginnote{Да, да, всё верно, временной ряд — это не ряд, ноль — чётное число,
единица — не простое, а Деда Мороза не существует.} случайных величин. 

Индекс временного ряда может быть бесконечным в обе стороны, 
\[
\ldots, y_{-2}, y_{-1}, y_0, y_1, y_2, \ldots  
\]
бесконечным в одну сторону,
\[
y_1, y_2, y_3, y_4, \ldots  
\]
либо конечным,
\[
y_1, y_2, y_3, y_4, \ldots, y_T.
\]
\end{definition}

Чтобы отличать весь временной ряд от одной конкретной случайной величины, мы будем использовать обозначения:

$y_t$ — одна конкретная случайная величина;

$(y_t) = y_1, y_2, y_3, y_4, \ldots$ — вся последовательность случайных величин.

Если контекст требует, то можно проявить больше аккуратности и указать возможные значения индекса, например, 
$(y_t)_{t = 1}^{\infty}$.


Начнём с самого простого временного ряда — белого шума.

\begin{definition}
Ряд $(u_t)$ называется белым шумом (white noise), если он удовлетворяет трём свойствам:

\begin{enumerate}
  \item Нулевое математическое ожидание, $\E(u_t) = 0$ для любого $t$.
  \item Постоянная дисперсия, $\Var(u_t) = \sigma^2_u$ для любого $t$.
  \item Нулевая ковариация, $\Cov(u_t, u_s) = 0$ для любых $t\neq s$. 
\end{enumerate}

\end{definition}


Заметим, что случайные величины в белом шуме вполне могут быть зависимы. Например, 

...




\begin{definition}
Ряд $(y_t)$ называется слабо стационарным (weekly stationary), или просто стационарным, 
если он удовлетворяет трём свойствам:

\begin{enumerate}
  \item Постоянное математическое ожидание, $\E(y_t) = \mu$ для любого $t$.
  \item Постоянная дисперсия, $\Var(y_t) = \gamma_0$ для любого $t$.
  \item Ковариация двух величин зависит только от их удалённости по времени друг от друга, $\Cov(y_t, y_s) = \gamma_{t-s}$ для любых $t$ и $s$. 
\end{enumerate}

\end{definition}
  




\begin{definition}
Ряд $(y_t)$ называется процессом скользящего среднего порядка $q$ (moving average of order $q$), 
если он представим в виде:

\[
y_t = u_t + b_1 u_{t-1} + \ldots b_q u_{t-q},
\]
где $(u_t)$ — белый шум. Обозначаем такие процессы мы так: $y_t \sim MA(q)$.
\end{definition}
  
\chapter{R — введение}










\Closesolutionfile{solution_file}





\section{Решения}


% для гиперссылок на условия
% http://tex.stackexchange.com/questions/45415
\renewenvironment{solution}[1]{%
         % add some glue
         \vskip .5cm plus 2cm minus 0.1cm%
         {\bfseries \hyperlink{problem:#1}{#1.}}%
}%
{%
}%


\input{solution_file}


\listoftodos

\section{Источники мудрости}

Источники мудрости, кои авторы постарались не замутить.
Смело направляйте к ним верблюдов своего любопытства!

\nocite{*}

\printbibliography


\end{document}
